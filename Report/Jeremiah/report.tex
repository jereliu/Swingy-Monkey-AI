\documentclass[11pt]{article}

%formating author affiliation
\usepackage{authblk}
\author[1]{(Jeremiah) Zhe Liu}
\author[2]{(Vivian) Wenwan Yang}
\author[1]{Jing Wen}
\affil[1]{Department of Biostatistics, Harvard School of Public Health}
\affil[2]{Department of Computational Science and Engineering, SEAS}

% change document font family to Palatino, and code font to Courier
\usepackage{mathpazo} % add possibly `sc` and `osf` options
\usepackage{eulervm}
\usepackage{courier}
%allow formula formatting

%identation in nested enumerates
\usepackage[shortlabels]{enumitem}
\setlist[enumerate,1]{leftmargin=1cm} % level 1 list
\setlist[enumerate,2]{leftmargin=2cm} % level 2 list

%flush align equations to left, this also loads amsmath 
%\usepackage[fleqn]{mathtools}
\usepackage{mathtools}
\usepackage{amsthm}
\DeclareMathAlphabet\mathbfcal{OMS}{cmsy}{b}{n}
\usepackage{comment}

%declare math symbolz
%# inner product
\DeclarePairedDelimiter{\inner}{\langle}{\rangle}

%declare argmin
\newcommand{\argmin}{\operatornamewithlimits{argmin}}
\newcommand{\argmax}{\operatornamewithlimits{argmax}}

%declare checkmark
\usepackage{pifont}% http://ctan.org/pkg/pifont
\newcommand{\cmark}{\ding{51}}%
\newcommand{\xmark}{\ding{55}}%

%title positon
\usepackage{titling} %fix title
\setlength{\droptitle}{-6em}   % Move up the title 

%change section title font size
\usepackage{titlesec} 
\titleformat{\section}
  {\normalfont\fontsize{12}{15}}{\thesection}{1em}{}
\titleformat{\subsection}
  {\normalfont\fontsize{12}{13}}{\thesubsection}{1em}{}
\titleformat{\subsubsection}
  {\normalfont\fontsize{12}{13}}{\thesubsubsection}{1em}{}

%overwrite bfseries to allow formula in section title  
\def\bfseries{\fontseries \bfdefault \selectfont \boldmath}

% change page margin
\usepackage[margin=0.8 in]{geometry} 

%disable indentation
\setlength\parindent{0pt}

%allow inserting multiple graphs
\usepackage{graphicx}
\usepackage[skip=1pt]{subcaption}
\usepackage[justification=centering,font=small]{caption}
\newcommand{\indep}{\rotatebox[origin=c]{90}{$\models$}}%indep sign

%allow code chunks
\usepackage{listings}
%\lstset{basicstyle=\footnotesize\ttfamily,breaklines=true}
\lstset{basicstyle=\footnotesize\ttfamily,breaklines=true}
\lstset{frame=lrbt,xleftmargin=\fboxsep, xrightmargin=-\fboxsep}
\lstset{language=R, commentstyle=\bfseries, 
keywordstyle=\ttfamily} %R-related formatting
\lstset{escapeinside={<@}{@>}}

%allow merged cell in tables
\usepackage{multirow}

%allow http links
\usepackage{hyperref}

%allow different font colors
\usepackage{xcolor}

%Thm and Def environment
\theoremstyle{definition}
\newtheorem{theorem}{Theorem}[section]
\newtheorem{lemma}[theorem]{Lemma}
\newtheorem{proposition}[theorem]{Proposition}
\newtheorem{corollary}[theorem]{Corollary}
\newtheorem{definition}[theorem]{Definition}

\newenvironment{definition2}[1][Definition]{\begin{trivlist} %def without index
\item[\hskip \labelsep {\bfseries #1}]}{\end{trivlist}}

\newenvironment{example}[1][Example]{\begin{trivlist} %def without index
\item[\hskip \labelsep {\bfseries #1}]}{\end{trivlist}}


%macros from Bob Gray
\usepackage{"./macro/GrandMacros"}
\usepackage{"./macro/Macro_BIO235"}

\begin{document}
%%%%%%%%%%%%%%%%%%%%%%%%%%%%%%%%%%%%%%%%%%%%%
%%%%%%%%%%%% TItle page with contents %%%%%%%%%%%%%%%
%%%%%%%%%%%%%%%%%%%%%%%%%%%%%%%%%%%%%%%%%%%%%

\title{\textbf{CS 181 Machine Learning}\\ 
\textbf{Practical 4 Report, Team \textit{la Derni\`{e}re Dame M}}}

\pretitle{\begin{centering}\Large}
\posttitle{\par\end{centering}}

\date{\today}
\vspace{-10em}
\maketitle
\vspace{-2em}


%%%%%%%%%%%%%%%%%%%%%%%%%%%%%%%%%%%%%%%%%%%%%
%%%%%%%%%%%% Formal Sections %%%%% %%%%%%%%%%%%%%%
%%%%%%%%%%%%%%%%%%%%%%%%%%%%%%%%%%%%%%%%%%%%%


\section{\textbf{Problem Description}}

Set in a \textit{Flappy Bird}-type game \textit{Swingy Monkey}, our current learning goal is to estimate an optimal policy $\pi: \Ssc \rightarrow \Asc$ such that the expectation of reward function $f: \Ssc \times \Asc \rightarrow \Rsc$ is maximized, i.e. we aim to identify a $\pi^*$ such that

\begin{align*}
\pi^* = arg\max_\pi E \Big( f(s, \pi(s)) | s \Big)
\end{align*}




State Space
Action Space


Empirical Goal: Score
Reward

\section{\textbf{Method}}

\subsection{\textbf{Rationale on Model Choice}}

\subsubsection{State Reduction and Discretization}

\subsubsection{Exploration/Exploitation Parameters}

Learning rate

$\epsilon$-greedy

\section{\textbf{Result}}

\subsection{\textbf{State Exploration}}

\subsection{\textbf{Convergence Behavior}}


\section{\textbf{Discussion \& Possible Directions}}


\newpage
\section*{\textbf{Reference}}
\begin{enumerate}
\item \label{ref:handbook}
Ricci F, Rokach L, Shapira B et al. (2010) \textbf{Recommender Systems Handbook}. \textit{Springer}. 
\item \label{ref:MFieee}
Koren Y, Bell R, Volinsky C. (2009) \textbf{Matrix factorization techniques for recommender systems}. \textit{IEEE Computer} Aug 2009, 42-49. 
\item \label{ref:WLA}
Srebro N,  Jaakkola T.(2003) \textbf{Weighted low-rank approximations}. \textit{Proceedings of the Twentieth International Conference} 720–727.
\item \label{ref:PMF}
R Salakhutdinov, A Mnih. (2008) \textbf{Probabilistic Matrix Factorization}. \textit{Advances in Neural Information Processing Systems} Vol. 20

\item \label{ref:implicit}
Koren, Y. (2008) \textbf{Factorization Meets the Neighborhood: a Multifaceted Collaborative Filtering Model}, \textit{Proc. 14th ACM SIGKDD International Conference on Knowledge Discovery and
Data Mining}.

\end{enumerate}

\end{document}

%%%%%%%%%%%%%%%%%%%%%%%%%%%%%
%%%%%%%%%%%%%%%%%%%%%%%%%%%%%
%%%%%%%%%%%%%%%%%%%%%%%%%%%%%
%%%%%%%%%%%%%%%%%%%%%%%%%%%%%
%%%%%%%%%%%%%%%%%%%%%%%%%%%%%
%%%%%%%%%%%%%%%%%%%%%%%%%%%%%
%%%%%%%%%%%%%%%%%%%%%%%%%%%%%
%%%%%%%%%%%%%%%%%%%%%%%%%%%%%
